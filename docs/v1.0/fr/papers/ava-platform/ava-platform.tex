\documentclass[runningheads]{llncs}

\usepackage[margin=1in]{geometry}
\usepackage{amsmath}
\usepackage{amssymb}
\usepackage{mathrsfs}
\usepackage{bbding}

\usepackage{lipsum}
\usepackage{comment}
\usepackage{graphicx}
\usepackage[table,xcdraw,pdftex,dvipsnames]{xcolor}
\usepackage{framed}
\setlength\FrameSep{1em}
\setlength\OuterFrameSep{\partopsep}
% \usepackage{authblk}
\usepackage[modulo]{lineno}
\linenumbers
\usepackage{xargs}                      % Use more than one optional parameter in a new commands
\usepackage[colorinlistoftodos,prependcaption,textsize=tiny]{todonotes}
\newcommandx{\XXXunsure}[2][1=]{\todo[linecolor=red,backgroundcolor=red!25,bordercolor=red,#1]{#2}}
\newcommandx{\XXXchange}[2][1=]{\todo[linecolor=blue,backgroundcolor=blue!25,bordercolor=blue,#1]{#2}}
\newcommandx{\XXXinfo}[2][1=]{\todo[linecolor=OliveGreen,backgroundcolor=OliveGreen!25,bordercolor=OliveGreen,#1]{#2}}
\newcommandx{\XXXimprovement}[2][1=]{\todo[linecolor=Plum,backgroundcolor=Plum!25,bordercolor=Plum,#1]{#2}}
\newcommandx{\XXX}[2][1=]{\todo[disable,#1]{#2}}
\usepackage{fancyhdr}
\newcommandx{\AVATokenName}{\texttt{\$AVAX}}
\newcommandx{\AVAPlatformName}{\texttt{Avalanche}}
\newcommandx{\AVAPlatformNameFirstRelease}{\texttt{Avalanche Borealis}}
\newcommandx{\genericAvalanche}{$\mathtt{A}^{*}$}
\usepackage[yyyymmdd,hhmmss]{datetime}

\setlength{\parindent}{2.0em}
\setlength{\parskip}{1.0em}
\renewcommand{\baselinestretch}{1.25}

\fboxsep=1pt%padding thickness
\fboxrule=0.2pt%border thickness

\begin{document}

\immediate\write18{git rev-parse HEAD > /tmp/temp.dat}

\title{La plateforme \AVAPlatformName{}\\\today}
\author{Kevin Sekniqi, Daniel Laine, Stephen Buttolph, and Emin G{\"u}n Sirer}
\institute{}

\maketitle

\begin{abstract}
Ce papier présente une vue globale de l'architecture de la première version de la plateforme \AVAPlatformName{}, dont
le nom de code est \AVAPlatformNameFirstRelease{}. Pour plus de détails sur les caractéristiques économiques de ce
jeton natif, nommé \AVATokenName{}, nous proposons au lecteur de se référer au papier~\cite{avatokenpaper} associé qui
détaille la dynamique de ce jeton.

\underline{Avertissement:} Les informations décrites dans ce papier le sont à titre préliminaire et peuvent être
amenées à changer ultérieurement. En outre, ce papier peut contenir des "déclarations anticipées."
\footnote{Ces déclarations anticipées se réfèrent généralement à des évènements à venir ou des performances futures.
Cela inclut, mais n'est pas limité à, la performance attendue d'Avalanche; le développement prévu de son entreprise et
de ses projets; l'exécution de sa vision et sa stratégie de croissance; et l'achèvement de projets actuellement en
cours de développement ou d'étude. Ces déclarations anticipées reflètent les croyances de notre équipe dirigeante
basées sur des hypothèses réalisées à la date de rédaction de ce document. Ces déclarations ne garantissent aucunement
les performances futures réelles de la plateforme, et ne doivent pas servir de base à des conclusions injustifiées. De
telles déclarations impliquent nécessairement des risques connus et inconnus, qui peuvent conduire à des résultats et
performances réelles futures sensiblement différents des projections exprimées explicitement ou implicitement dans ce
document. Avalanche ne s'engage à aucune obligation de mettre à jour ces déclarations anticipées. Bien que celles-ci
représentent nos meilleures prédictions au moment où nous les avons faites, il ne peut y avoir aucune assurance qu'elles
s'avèrent exactes vu que les résultats réels et évènements futurs sont amenés à les contredire sensiblement. Le lecteur
est mis en garde de ne pas placer d'attentes injustifiées sur ces déclarations anticipées.}

\end{abstract}
\begin{center}
    \scriptsize Git Commit: \input{/tmp/temp.dat}
 \end{center}
\section{Introduction}
\label{section:introduction}
Ce papier présente une vue globale de la plateforme \AVAPlatformName{}. Il se focalise principalement sur les 3 éléments
différenciateurs de la plateforme: le moteur, le modèle d'architecture, et le mécanisme de gouvernance.

\subsection{\AVAPlatformName{} Objectifs et principes}
\AVAPlatformName{} est une plateforme basée sur une blockchain hautement performante, évolutive, paramétrable et fiable.
Elle vise trois cas d'utilisation généraux:
\begin{itemize}
\item Construire des blockchains spécifiques à une application donnée, couvrant les déploiements avec (privés) ou sans
  (publics) contrôle d'accès.
\item Construire et exécuter des applications hautement évolutives et décentralisées (Dapps).
\item Construire des actifs numériques complexes de manière arbitraire permettant la personnalisation des règles,
  des clauses et des avenants (actifs intelligents).
\end{itemize}
L'objectif global d'\AVAPlatformName{} est de fournir une plateforme unificatrice pour la création, le transfert, et
l'échange d'actifs numériques.

\noindent De par sa construction, \AVAPlatformName{} possède les propriétés suivantes:

\paragraph{Evolutive} \AVAPlatformName{} est conçue pour passer massivement à l'échelle, et être robuste et efficace.
Le consensus principal est capable de supporter un réseau global jusqu'à un nombre potentiel de centaines de millions
d'appareils connectés à internet plus ou moins puissants, qui opèrent de manière fluide avec des latences faibles et un
très grand nombre de transactions par seconde.

\paragraph{Fiable} \AVAPlatformName{} est conçue pour être robuste et permettre un haut niveau de sécurité. Les
protocoles de consensus classiques sont conçus pour contenir jusqu'à $f$ attaquants, et échoucher entièrement devant
un attaquant de taille $f+1$ ou plus. Le consensus de Nakamoto ne garantit plus aucune sécurité lorsque 51\% des
mineurs sont byzantins. Par contraste, \AVAPlatformName{} permet une très forte garantie de sécurité lorsque l'attaquant
est en dessous d'un certain seuil configurable par le concepteur du système, et suit une dégradation élégante lorsque
l'attaquant dépasse ce seuil. Elle peut maintenir des garanties de sécurité (mais pas la vitalité) même quand
l'attaquant dépasse 51\% d'influence. C'est le premier système sans contrôle d'accès qui fournit des garanties de
sécurité si fortes.

\paragraph{Décentralisée} \AVAPlatformName{} est conçue pour assurer un niveau de décentralisation sans précédent. Cela
implique un engagement envers plusieurs implémentations clientes et aucun contrôle centralisé de quelque sorte.
L'écosystème est conçu pour éviter les divisions entre classes d'utilisateurs ayant différents intérêts. Mais surtout
il n'y a pas de distinction entre les mineurs, les développeurs, et les utilisateurs

\paragraph{Gouvernable et démocratique} \AVAPlatformName{} est une plateforme hautement inclusive, qui permet à
n'importe qui de se connecter à son réseau, participer dans la validation et être directement impliqué dans la
gouvernance. Tous les détenteurs de jetons ont un droit de vote pour la sélection des paramètres financiers principaux
et le choix dans la manière dont le système évolue.

\paragraph{Interopérable et flexible} \AVAPlatformName{} est conçue pour être une infrastructure universelle et flexible
hébergeant une multitude d'actifs et de blockchains, où l'actif de base \AVATokenName{} est utilisé pour sa sécurité
et comme une unité de compte permettant les échanges. Le système est prévu pour supporter, de manière neutre vis à vis
de la valeur, de nombreuses blockchains pouvant être construites sur sa base. La plateforme est conçue à partir de zéro
pour y faciliter le portage de blockchains existantes, l'import de comptes, le support de multiple langages de scripts
et de machines virtuelles, et supporter de multiples scénarios de déploiement de manière significative.

\subsubsection{Plan}
Le reste de ce papier est séparé en quatre sections majeures. La section~\ref{section:engine} résume les détails du
moteur qui alimente la plateforme. La section~\ref{section:platform_overview} discute du modèle d'architecture derrière
la plateforme, en incluant les sous-réseaux, les machines virtuelles, l'amorçage, la participation, et la mise en jeu.
La section~\ref{section:governance_and_token} explique le modèle de gouvernance qui permet le changement dynamique des
paramètres économiques principaux. Finalement, la section~\ref{section:governance_and_token} explore différents points
d'intérêt annexes, comprenant les optimisations potentielles, la cryptographie post-quantique, et les adversaires
potentiels.

\subsubsection{Convention de nommage}
Le nom de la plateforme est \AVAPlatformName{}, et est généralement mentionnée comme
``la plateforme \AVAPlatformName{}'', et peut être synonyme du ``réseau \AVAPlatformName{}'', ou tout simplement
\AVAPlatformName{}. Le code source sera versionné selon trois identifiants numériques, labelisé
``\texttt{v.[0-9].[0-9].[0-100]}'', où le premier numéro identifie les livraisons majeures, le deuxième identifie les
livraisons mineures, et le troisième numéro les patchs. La première livraison publique, dont le nom de code est
\AVAPlatformNameFirstRelease{}, est nommée \texttt{v. 1.0.0}. La famille de protocoles de consensus utilisés par la
plateforme \AVAPlatformName{} est mentionnée comme étant la famille Snow*. Il en existe trois instances concrètes,
appelées Avalanche, Snowman, et Frosty.

\section{Le moteur}
\label{section:engine}
Les détails à propos de la plateforme \AVAPlatformName{} commencent avec le composant principal qui alimente la
plateforme: le moteur de consensus.

\subsubsection{Contexte}
Les paiements et -- plus généralement -- les calculs distribués, requièrent un accord entre un ensemble de machines. En
conséquent les protocoles de consensus, qui permettent à un groupe de noeuds d'arriver à un accord, se trouvent au coeur
des blockchains ainsi que dans presque tous les systèmes distribués industriels déployés à grande échelle. Le sujet a
été largement approfondi depuis presque 50 ans, et cet effort à ce jour n'a fait émerger que deux familles de
protocoles: les protocoles de consensus classiques qui se basent sur une communication entre tous les pairs, et le
consensus de Nakamoto qui s'appuie sur le minage par preuve de travail couplé à la règle de la plus longue chaîne.
Tandis que les protocoles de consensus classiques peuvent avoir une latence faible et un débit élevé, ils ne passent
pas à l'échelle sur un grand nombre de participants et ne sont pas non plus robustes en la présence de changements de
participants, ce qui les a relégué au rang de systèmes avec contrôle d'accès déployés statiquement. Les protocoles de
consensus de Nakamoto~\cite{nakamoto2008bitcoin,wood2014ethereum,EyalGSR16} en revanche sont robustes mais souffrent
de latences de confirmation élevées, d'un faible débit, et demandent une dépense d'énergie constante pour assurer leur
sécurité.

La famille de protocoles Snow*, introduite par Avalanche, combine les meilleures propriétés des protocoles de consensus
classiques avec le meilleur de ceux de Nakamoto. Basés sur un mécanisme d'échantillonnage léger, ils obtiennent une
latence faible et un débit élevé sans avoir besoin de s'accorder précisément sur les particpants du système.
Ils passent facilement à l'échelle de milliers à des millions de participants agissant directement sur le protocole de
consensus. De plus, ces protocoles ne font pas usage du minage par preuve de travail et s'affranchissent de surcroît
de besoins exhorbitants en énergie et de la fuite de valeur inhérente dans l'écosystème, amenant à des protocoles qui
sont légers, écologiques, et latents.

\subsubsection{Mécanisme et propriétés}
Les protocoles Snow* opèrent en échantillonnant le réseau de manière répétée. Chaque node sonde un ensemble réduit et de
taille constante de voisins choisis aléatoirement, et change son état si une supermajorité porte une valeur différente.
L'échantillonnage est répété jusqu'à converger vers une valeur unique, ce qui arrive rapidement en fonctionnement
normal. On peut illustrer ce fonctionnement par un exemple concret. Tout d'abord une transaction est créée par un
utilisateur et envoyée à un noeud de validation, qui est un noeud participant dans la procédure de consensus. Elle est
ensuite propagée à tous les autres noeuds du réseau via un protocole dit de "bavardage" (gossip). Que se passe-t-il si
cet utilisateur envoie également une transaction conflictuelle, à savoir une double dépense? Afin de choisi parmi les
deux transactions en conflit et éviter la double dépense, chaque noeud va choisir de manière aléatoire un petit
sous-ensemble de noeuds et leur demander laquelle de ces transactions ils pensent être la transaction valide. Si le
noeud à l'origine de la requête reçoit une réponse super-majoritaire en faveur d'une transaction, alors le noeud change
sa propre réponse pour cette même transaction. Chaque noeud du réseau répète cette procédure jusqu'à ce que le réseau
entier parvienne à un consensus sur l'une des transactions conflictuelles.

Etonnamment, bien que le mécanisme principal de fonctionnement demeure assez simple, ces protocoles conduisent à une
dynamique système très intéressante qui les rendent adaptés à un déploiement à grande échelle.
\begin{itemize}
\item \textit{Sans contrôle d'accès, ouverts au roulement, et robustes}. Les derniers nés des projets blockchains
emploient des protocoles de consensus classiques et doivent par conséquent avoir une connaissance complète des
participants du réseau. Connaître l'ensemble des participants est suffisamment simple dans le milieu fermé des systèmes
avec contrôle d'accès, mais devient de plus en plus compliqué dans les réseaux ouverts et décentralisés. Cette
limitation impose des risques de sécurité élevés aux projets en place qui emploient de tels protocoles. Par
contraste, les protocoles Snow* maintiennent des garanties de sécurité élevées même lorsqu'il existe des divergences
bien quantifiées sur le réseau du point de vue de deux noeuds différents. Les validateurs des protocoles Snow* jouissent
de leur capacité à effectuer la validation sans connaissance assidue et continue des membres du réseau. Il sont donc
robustes et largement adaptés aux blockchains publiques.
\item \textit{Evolutifs et décentralisés}. Une des fonctionnalités clé de la famille Snow est sa capacité à passer à
l'échelle sans entraîner de compromis sur les fondamentaux. Les protocoles Snow peuvent atteindre jusqu'à des dizaines
de milliers ou des millions de noeuds, sans avoir à déléguer vers des sous-ensemble de validateurs. Ces protocoles
jouissent d'une décentralisation système de premier ordre, permettant à chaque noeud d'opérer une validation complète.
Une participation de premier plan et continue a des impacts profonds pour la sécurité du système. Dans quasiment tous
les protocoles à preuve d'enjeu qui essayent de passer à l'échelle d'un grand nombre de participants, le mode opératoire
typique consiste à permettre le passage à l'échelle en délégant la validation à un sous-comité. Naturellement cela
implique que la sécurité du système est précisément aussi élevée que le coût nécessaire à la corruption du sous-comité.
Les sous-comités sont de plus sujets à la formation de cartels.

Dans les protocoles de type Snow une telle délégation n'est pas nécessaire, ce qui permet à chaque opérateur d'un noeud
d'avoir une influence de premier ordre dans le système à tout moment. Une autre conception, typiquement appelée
"sharding d'état", s'efforce de fournir une meilleur évolutivité en parallélisant le traitement en série des transactions
sur plusieurs réseaux indépendants de validateurs. Malheureusement la sécurité du système dans une telle architecture
devient au maximum équivalente au sous-réseau indépendant le plus facilement corruptible. En définitive, ni l'élection
de sous-comités ni le sharding ne sont adaptés à une stratégie de mise à l'échelle de plateformes crypto.
\item \textit{Adaptatifs}. Contrairement à d'autres systèmes basés sur le vote, les protocoles Snow* atteignent de bien
meilleures performances quand les adversaires sont peu nombreux, et pourtant demeurent hautement résistants lorsqu'ils
subissent des attaques plus larges.
\item \textit{Sûrs de manière asynchrone}. Les protocoles Snow*, contrairement aux protocoles dits "de plus longue
chaîne", ne requièrent pas de synchronicité stricte pour opérer de manière sûre et par conséquent empêchent les doubles
dépenses même face à un partitionnement du réseau. Avec Bitcoin par exemple, si l'hypothèse de synchronicité est violée
il devient possible d'opérer deux branches du réseau Bitcoin pour de longues périodes de temps, ce qui aurait pour
conséquence d'invalider des transactions au moment de consolider les branches.
\item \textit{Faible latence}. La plupart des blockchains aujourd'hui sont incapables de supporter des applications
métier, comme le trading ou les paiements de biens quotidiens. Il n'est simplement pas envisageable d'attendre plusieurs
minutes, ou même des heures, pour la confirmation de transactions. En conséquence l'une des plus importantes, et pourtant
largement négligée, propriétés des protocoles de consensus est le temps de finalisation. Les protocoles Snow* atteignent
typiquement la finalité d'une transaction en $\leq 1$ seconde, ce qui est significativement plus faible que pour les
protocoles de plus longue chaîne et les blockchains utilisant le sharding, pour qui cette finalité est typiquement une
affaire de minutes.
\item \textit{Haut débit}. Les protocoles Snow*, qui peuvent construire une chaîne linéaire ou un DAG (Graphe Acyclique
Dirigé), atteignent des milliers de transactions par seconde (5000+ tps) tout en conservant un niveau de décentralisation
totale. Les nouvelles solutions en matière de blockchain qui revendiquent un grand nombre de transactions par seconde
le font au dépens de la décentralisation et la sécurité et optent pour des mécanismes de consensus plus centralisés et
risqués. Certains projets affichent des chiffres provenant d'environnements extrêmement contrôlés, qui maquillent donc
les véritables résultats de performance. Les chiffres affichés pour \AVATokenName{} sont tirés directement d'un réseau
\AVAPlatformName{} réel entièrement opérationnel de 2000 noeuds sur AWS, distribués équitablement à travers le globe et
tournant sur des machines peu coûteuses. Des résultats de performance encore plus élevés peuvent être obtenus en prenant
l'hypothèse d'une meilleure bande passante allouée à chaque noeud et du matériel dédié à la vérification des signatures.
Finalement, nous notons que les métriques sus-mentionnées se situent au niveau de la couche de base. Des solutions de
mise à l'échelle de niveau 2 pourraient améliorer ces résultats de manière considérable.
\end{itemize}

\newpage

\subsubsection{Tableau comparatif des consensus}
Le Tableau \ref{table:comparativechartconsensus} décrit les différences entre les trois familles de protocoles de
consensus connues à travers 8 axes critiques:

\begin{table}[h!]
\centering
\begin{tabular}{l|ccc}
& \ \ \ \ \ Nakamoto\ \ \ \ &\  \ \ \ \  Classical\ \ \ \ \  & \ \ \ \ Snow* \ \ \ \ \\ \hline
\rowcolor[HTML]{EFEFEF}
Robuste (Adapté aux configurations ouvertes)                             & +        & -         & +     \\
Hautement décentralisé (permet un grand nombre de validateurs)           & +        & -         & +     \\
\rowcolor[HTML]{EFEFEF}
Faible latence et finalité rapide (Confirmation rapide des transactions) & -        & +         & +     \\
Haut débit (supporte un grand nombre de clients)                         & -        & +         & +     \\
\rowcolor[HTML]{EFEFEF}
Léger (Spécifications matérielles modestes)                              & -        & +         & +     \\
Latent (Inactif lorsqu'aucune décision n'est nécessaire)                 & -        & +         & +     \\
\rowcolor[HTML]{EFEFEF}
Sécurité paramétrable (A partir de 51\% d'adversaires présents)          & -        & -         & +     \\
Hautement évolutif                                                       & -        & -         & +
\end{tabular}
\caption{Tableau comparatif entre les trois familles de protocoles de consensus connues. Avalanche, Snowman, et Frosty
appartiennent tous à la famille Snow*.}
\label{table:comparativechartconsensus}
\end{table}

\section{Vue d'ensemble de la plateforme}
\label{section:platform_overview}
Dans cette section nous exposons une vue d'ensemble de la plateforme et discutons certains détails d'implémentation. La
plateforme \AVAPlatformName{} sépare clairement trois préoccupations: les chaînes (et les actifs qui reposent dessus),
les environnements d'exécution, et le déploiement.

\subsection{Architecture}

\subsubsection{Sous-réseaux}
L'architecture de l'internet moderne est décrite comme une architecture dite ``à taille étroite'', où il y a une couche
protocolaire de communication unique située au milieu, composée des protocoles HTTP/TLS/IP/TCP, avec une matrice de
réseaux très étendue en dessous et les applications au dessus. Une telle conception n'était pas prévue, mais émergea
néanmoins au fur et à mesure de l'évolution d'internet vers des fonctions bien définies. Un tel modèle d'architecture
est en contraste total avec la manière dont les plateformes à base de blockchain ont été développées jusqu'à
maintenant. Presque toutes les plateformes à base de blockchain suivent une architecture monolithique, typiquement
inspirée du modèle Bitcoin. Naturellement cela présente un ensemble de limitations importantes qui empêchent les
plateformes à base de blockchain de couvrir un grand nombre de cas d'usage.

\AVAPlatformName{} est conçue selon le même modèle d'architecture qu'internet, et donc de manière à être extensible,
modulaire, et composable. La plateforme \AVAPlatformName{} a de multiples sous-réseaux logiquement séparés, supportant
chacuns leurs propres environnements d'exécution, ensemble de règles, et validateurs. Lorsque l'on créé un nouveau
sous-réseau, on peut spécifier une multitude de paramètres, incluant -- mais non limité à -- les suivants:
\begin{itemize}
\item{} Qui peut proposer et valider les blocs ?
\item{} De quoi un bloc valide est-il constitué?
\item{} L'état de génèse de la chaîne.
\item{} Quelle transition d'état s'effectue lorsqu'un bloc est accepté?
\item{} Quels appels RPC sont supportés?
\item{} Quoi sauver dans la base de données, et quand?
\end{itemize}
L'architecture en sous-réseaux permet en somme à n'importe qui de déployer sa propre blockchain correspondant à ses
besoins particuliers. Sur la plateforme \AVAPlatformName{}, tout est modélisable sous la forme d'un sous-réseau, tout
comme le jeton natif \AVATokenName{} qui réside sur son propre sous-réseau. Le sous-réseau supportant le jeton
\AVATokenName{}, comportant le label ID-0 est spécial de par les propriétés suivantes:
\begin{enumerate}
\item Chaque validateur dans le réseau \AVAPlatformName{} est également validateur du sous-réseau émettant le jeton
\AVATokenName{}.
\item Chaque transaction émise entre deux sous-réseaux, comme les transferts d'actifs, sont gérés par le sous-réseau du
jeton \AVATokenName{}.
\item La création d'un nouveau sous-réseau nécessite de payer des frais en jetons \AVATokenName{}, ce qui est nécessaire
pour pouvoir modifier le sous-réseau de "staking".
\end{enumerate}
Notons que les sous-réseaux ne sont pas des réseaux isolés. Ils sont entièrement intéropérables avec le reste du réseau,
ce qui veut dire qu'ils peuvent échanger avec d'autres sous-réseaux de manière atomique.

\subsubsection{Machines Virtuelles}
Dans le but de lancer un nouveau sous-réseau, un développeur doit d'abord écrire le code de la machine virtuelle (ou VM)
correspondante. Les machines virtuelles dessinent les plans d'un sous-réseau à déployer sur \AVAPlatformName{}.
Lorsqu'il est prêt à créer un sous-réseau, le développeur spécifie la VM à utiliser ainsi que l'état de génèse de
celle-ci. Le comportement de la chaîne ou le DAG d'un sous-réseau donné est entièrement défini par la machine virtuelle
que ce sous-réseau utilise. Naturellement vu qu'un sous-réseau n'est qu'une instance d'une VM, il peut y avoir autant de
réseaux que possible qui utilisent la même VM. Cependant, même si plusieurs sous-réseaux peuvent utiliser la même VM,
ils maintiennent chacun leur propre état qui est indépendant de l'état des autres sous-réseaux. Dans un but de
simplification, nous écrivons parfois qu'un sous-réseau ``utilise'' une VM ou qu'un sous-réseau ``exécute'' une VM.
Dans les deux cas nous cherchons à dire que le sous-réseau est une instance de cette machine virtuelle.

\AVAPlatformName{} supporte le développement d'un marché riche en machines virtuelles qui profite d'une conformité
stricte avec le jeu de règles, la confidentialité, et d'autres nouvelles fonctionnalités innovantes. Un tel dépôt
flexible de briques réutilisables permet aux développeurs amateurs mais aussi aux grandes entreprises de déployer des
applications système et métier entièrement conformes et intéropérables. Le jeton \AVATokenName{} représente
l'infrastructure principale qui garantit la sécurité du système tout en fournissant une unité d'échange universelle
entre tous les actifs deployés sur \AVAPlatformName{}.

\subsection{Amorçage}
La première étape pour participer à la plateforme \AVAPlatformName{} est l'amorçage. Ce procédé se déroule en trois
étapes: la connection à des points d'ancrage, la mise en réseau et la découverte de l'état, et le passage en tant
que validateur.

\subsubsection{Points d'ancrage}
Tout système constitué d'un réseau de pairs qui opère sans un jeu d'identités régies par des permissions (qui sont
codées en dur) nécéssite un mécanisme de \emph{découverte de pairs}. Dans les réseaux de partage de fichiers pair-à-pair
un ensemble de trackers est utilisé. Dans les réseaux de cryptomonnaies, on utilise généralement un mécanisme de noeuds
d'ancrage DNS (que l'on appellera points d'ancrage) qui contiennent un jeu prédéfini d'adresses IP d'ancrage à partir
desquelles les autres membres du réseau peuvent être découverts. Le rôle des noeuds d'ancrage DNS est de fournir des
informations utiles à propos de l'ensemble des participants actifs du système. Le même mécanisme est utilisé dans
Bitcoin Core~\cite{bitcoin_2018}, dont le code source comprend le fichier \texttt{src/chainparams.cpp} qui contient
une liste de noeuds d'ancrage codée en dur. La différence entre BTC et \AVAPlatformName{} est que BTC ne nécessite
qu'un seul noeud d'ancrage DNS correct, là ou \AVAPlatformName{} a besoin d'une majorité simple de noeuds d'ancrage
étant corrects. Par exemple un nouvel utilisateur peut choisir d'amorcer sa vue du réseau à travers un ensemble
de places d'échanges réputées et bien établies \emph{sans} avoir à leur faire confiance à tous individuellement. Notons
cependant que la liste des noeuds d'amorçage ne nécéssite pas forcémment d'être statique ou codée en dur et peut être
renseignée par l'utilisateur, même si dans un souci de facilité d'utilisation les clients préfèreront fournir une liste
par défaut pointant vers des acteurs économiquement importants, comme des bourses d'échange de cryptomonnaies, avec
lesquels les clients souhaitent partager leur vision du monde. Il n'y a aucune barrière à devenir un noeud d'ancrage,
en conséquence un groupement de points d'ancrage ne peut pas dicter si un noeud peut ou ne peut pas rejoindre le
réseau vu que les noeuds peuvent décrouvrir le réseau de pairs \AVAPlatformName{} le plus récent en s'attachant à
n'importe quel groupe de noeuds d'ancrage.

\subsubsection{Mise en réseau et découverte de l'état}
Une fois connecté aux noeuds d'ancrage, un noeud lance une requête pour obtenir l'ensemble des transitions d'état
le plus récent. Nous appelons cet ensemble de transitions la \emph{frontière admise}. Pour une chaîne la frontière
admise est le dernier bloc accepté. Pour un DAG la frontière admise est l'ensemble des sommets qui sont acceptés mais
qui n'ont aucun enfant accepté. Après avoir récupéré les frontières admises retournées par les points d'ancrage, les
transitions d'état qui sont acceptées par une majorité de points d'ancrage sont définis comme acceptés. L'état correct
est donc extrait en se synchronisant avec les noeuds échantillonnés. Tant qu'il y a une majorité de noeuds corrects
dans l'ensemble des points d'ancrage, alors les transitions d'état acceptées doivent avoir été marquées comme
acceptées par au moins un noeud correct. Le procédé de découverte de l'état est également utilisé dans la découverte
du réseau. La liste des membres du réseau est définie dans la chaîne de validation. Par conséquent la synchronisation
avec la chaîne de validation permet au noeud de découvrir la liste des validateurs courants. La chaîne de validation
sera décrite plus en détails dans la prochaine section.

\subsection{Contrôle Sybil et appartenance au réseau}
Les protocoles de consensus fournissent des garanties de sécurité en partant de l'hypothèse qu'un seuil maximal de
membres du système peuvent être antagonistes. Une attaque Sybil lors de laquelle un noeud peut innonder le réseau
d'identités malicieuses à moindre coût peut facilement invalider ces garanties. Fondamentalement, une telle attaque peut
uniquement être contrée en faisant le compromis de la présence par la preuve d'une ressource difficile à
générer~\cite{douceur2002sybil}. Les anciens systèmes ont exploré l'utilisation de mécanismes de protection anti-Sybil
qui incluent la preuve de travail (PoW), la preuve d'enjeu (PoS), la preuve de temps écoulé (POET), la preuve d'espace
et temps (PoST), et la preuve d'autorité (PoA).

A leur coeur, tous ces mécanismes servent la même fonction: ils demandent à tous les participants d'avoir un ``intérêt
personnel''  sous la forme d'un engagement économique, qui fournit en retour une barrière économique contre la mauvaise
conduite de ce participant. Ils impliquent tous une forme d'enjeu, qu'il soit sous la forme de stations de minage et de
puissance de hachage (PoW), d'espace disque (PoST), de matériel certifié (POET) ou d'une identité approuvée (PoA). Cet
enjeu forme la base d'un coût économique que les participants doivent supporter pour acquérir une voix. Par exemple,
dans Bitcoin, la capacité à contribuer des blocs valides est directement proportionnelle à la puissance de hachage du
participant qui propose les blocs. Malheureusement il y a également une importante confusion entre les protocoles de
consensus et les mécanismes de contrôle Sybil. Nous notons que le choix des protocoles de consensus est, pour la
plupart, directement lié au choix du mécanisme de contrôle Sybil. Cela ne veut pas dire que les mécanismes de contrôle
Sybil peuvent être directement interchangeables, sachant qu'un choix particulier peut avoir des implications sur les
garanties sous-jacentes du protocole de consensus. En revanche, la famille Snow* peut être couplée avec beaucoup de ces
mécanismes connus sans nécessiter de modification importante.

En fin de compte, pour sa sécurité et s'assurer que les motivations des participants soient en phase pour le bien du
réseau, \AVATokenName{} a choisi la preuve d'enjeu comme mécanisme de contrôle Sybil principal. Certaines formes
d'enjeu sont intrinsèquement centralisées: la fabrication de stations de minage (PoW), par exemple, est par nature
centralisée entre les mains d'un nombre limité de personnes qui ont le savoir-faire adéquat et ont accès à la douzaine
de brevets requis pour pouvoir fabriquer des composants électroniques de manière compétitive. De plus le minage par
preuve de travail laisse s'écouler de la valeur à cause de l'énorme subvention annuelle des mineurs. De la même manière
l'espace disque est abondamment détenu par les grands opérateurs de data centers. En outre tous les mécanismes de
contrôle Sybil qui augmentent les coûts d'usage, comme l'électricité pour le hachage, transfèrent de la valeur en dehors
de l'écosystème sans même aborder le problème de la destruction de l'environnement. Ceci en retour réduit l'enveloppe de
rentabilité du jeton, pour lequel un mouvement de prix sur une petite période de temps peut rendre le système
inutilisable. La preuve de travail sélectionne par nature les mineurs qui ont les relations pour se procurer de
l'électtricité à moindre coût, ce qui n'a pas grand chose à voir avec la capacité du mineur à sérialiser des
transactions ou avec ses contributions à l'écosystème en général. Parmi ces options, nous choisissons la preuve d'enjeu
parce que c'est écologique, accessible, et ouvert à tous. Notons cependant que tandis que \AVATokenName{} utilise la
preuve d'enjeu, le réseau \AVAPlatformName{} permet le lancement de sous-réseau avec preuve de travail (PoW) et preuve
d'enjeu (PoS).

Le staking par enjeu est un mécanisme naturel pour la participation dans un réseau ouvert car il met en oeuvre un
argument économique direct: la probabilité de succès d'une attaque est directement proportionnelle à une fonction de
coût monétaire bien définie. En d'autres termes les noeuds qui prouvent leur enjeu sont motivés économiquement à ne pas
s'aventurer vers un comportement qui pourrait nuire à la valeur de ce qu'ils ont mis en jeu. De plus la somme mise en
jeu ne génère pas de coût additionnel de maintenance (autre que l'opportunité d'investir cette somme dans un autre
actif), et possède la propriété qu'elle soit, contrairement aux équipements de minage, entièrement détruite en cas
d'attaque catastrophique. Dans le cas de la preuve de travail, l'équipement de minage peut être simplement réutilisé ou
-- si son propriétaire en décide ainsi -- entièrement vendu sur le marché secondaire.

Un noeud souhaitant participer au réseau peut le faire librement en mettant en jeu une somme qui est immobilisée pendant
toute la durée de participation au réseau. L'utilisateur détermine la durée de cette mise en jeu. Une fois acceptée, une
période de mise en jeu ne peut pas être annulée. Le but principal de cette démarche est de s'assurer que les noeuds
partagent en substance la même vision stable du réseau. Nous pensons définir un temps de mise en jeu minimal de l'ordre
de la semaine.

Contrairement à d'autres systèmes qui proposent également un mécanisme de preuve d'enjeu, \AVATokenName{} ne fait pas
usage du ``slashing'' (ou punition), et par conséquent l'intégralité du montant mis en jeu est retourné lorsque la
période d'enjeu est écoulée. Cela évite les scénarios involontaires comme les pannes matérielles et logicielles qui
amènent à un perte de jetons. Ceci concorde avec notre philosophie de conception dans la construction de technologie
prévisible: les jetons mis en jeu ne sont pas en danger, même en la présence de failles logicielles ou matérielles.

Dans la plateforme \AVAPlatformName, un noeud qui veut participer envoie une \emph{transaction d'enjeu} spéciale à la
chaîne de validation. Les transactions d'enjeu définissent un montant à mettre en jeu, la clé d'enjeu du partcipant qui
souhaite mettre en jeu, la durée, et la date à partir de laquelle la participation commence. Une fois la transaction
acceptée, les fonds sont bloqués jusqu'à la fin de la période d'enjeu. Le montant minimal autorisé est décidé et
appliqué par le système. Le montant mis en jeu par un participant a des implications à la fois sur la quantité
d'influence que le participant a sur le procédé de consensus, et sur la récompense comme expliqué plus loin.
La période d'enjeu spécifiée doit être entre $\delta_{min}$ et $\delta_{max}$, les périodes minimales et maximales pour
lesquelles la somme mise en jeu est vérouillée quelqu'en soit le montant. Comme pour le montant mis en jeu, la période
a également des implications sur le montant de la récompense prévue par le système. La perte ou le vol de la clé
d'enjeu ne peut pas mener à une perte d'actif, sachant que cette clé est uniquement utilisée dans le procédé de
consensus et non pas pour le transfert d'actifs.

\subsection{Les contrats intelligents dans \AVATokenName}
A son lancement la plateforme \AVAPlatformName{} supporte les contrats intelligents standard basés sur le langage
Solidity à travers la machine virtuelle Ethereum (EVM). Nous envisageons que la plateforme supporte un ensemble plus
riche et plus puissant d'outils d'écriture de contrats, incluant:
\begin{itemize}
\item Contrats intelligents avec exécution hors-chaîne et vérification sur la chaîne.
\item Contrats intelligents à exécution parallèle. Tout contrat intelligent qui n'opère pas selon le même état dans tous
les sous-réseaux d'\AVAPlatformName{} pourra s'exécuter en parallèle.
\item Un version améliorée de Solidity, appelée Solidity++. Ce nouveau langage supportera le versionnage, les opérations
mathématiques sûres et l'arithmétique à virgule fixe, un typage amélioré, la compilation par LLVM, et l'exécution
"juste à temps" (JIT - Just In Time).
\end{itemize}

Si un développeur a besoin du support de l'EVM mais veut déployer des contrats intelligents dans un sous-réseau privé,
il peut directement démarrer un sous-réseau. C'est de cette manière qu'\AVAPlatformName{} permet le sharding spécifique
à une fonctionnalité à travers les sous-réseaux. De plus si un développeur a besoin d'intéragir avec les contrats
intelligents Ethereum actuellement déployés, il peut interagir avec le sous-réseau Athereum qui est dérivé d'Ethereum.
Pour finir, si un développeur a besoin d'un environnement d'exécution différent de la machine virtuelle Ethereum, il peut
choisir de déployer son contrat intelligent à travers un sous-réseau qui implémente un environnement d'exécution
différent comme DMAL ou WASM. Les sous-réseaux peuvent supporter des fonctionnalités additionnelles au delà du
fonctionnement d'une machine virtuelle. Par exemple les sous-réseaux peuvent imposer des contraintes de performances
pour les noeuds de validation les plus gros qui maintiennent des contrats intelligents pour des périodes plus longues,
ou des validateurs qui gardent l'état du contrat de manière privée.

\section{La gouvernance et le jeton \AVATokenName{}}
\label{section:governance_and_token}

\subsection{Le jeton natif \AVATokenName{}}
Le jeton natif, \AVATokenName{}, est limité en quantité, avec une limite définie à $720,000,000$ jetons, parmi lesquels
$360,000,000$ jetons sont disponibles au lancement du réseau principal. Cependant, contrairement à d'autres jetons en
quantité limitée émis selon un taux d'émission perpétuel, \AVATokenName{} est conçu pour réagir aux changements de
conditions économiques. En particulier, l'objectif de la politique monétaire d'\AVATokenName{} est d'équilibrer les
incitations des utilisateurs à mettre en jeu leurs jetons (staking) ou les utiliser pour intéragir avec la multitude de
services disponibles sur la plateforme. Les participants à la plateforme agissent de manière collective comme une banque
de réserve décentralisée. Les leviers disponibles sur Avalanche sont les récompenses de staking, les frais, et les
"parachutages" (airdrops) de jetons gratuits, qui sont tous influencés par des paramètres de gouvernance. Les
récompenses de staking sont définis par la gouvernance sur la blockchain, et sont régies par une fonction conçue pour
ne jamais dépasser la limite en quantité de jetons. L'incitation au staking peut se faire en augmentant les frais ou les
récompenses de mise en jeu. De l'autre côté on peut encourager un engagement vers les services de la plateforme
Avalanche en baissant les frais et les récompenses de staking.

\subsubsection{Cas d'usage}
\paragraph{Paiements}
Les paiements pair à pair réellement décentralisés sont encore un doux rêve pour l'industrie à cause du manque de
performance actuel de la part des protagonistes en place. \AVATokenName{} est aussi puissant et facile à utiliser que
les paiements Visa, permettant jusqu'à des milliers de transactions chaque seconde au niveau mondial, d'une manière
complètement décentralisée et sans tiers de confiance. En outre, pour les marchands à travers le monde, \AVATokenName{}
offre une proposition de valeur directe par rapport à Visa, à savoir des frais de transaction plus faibles.

\paragraph{Mise en jeu: sécuriser le système}
Sur la plateforme \AVAPlatformName{}, le contrôle des attaques sybil est obtenu via le staking. Pour pouvoir valider, un
participant doit bloquer une somme qui est mise en jeu. Les validateurs, parfois appelés "stakers", sont récompensés
pour leurs services de validation sur la base du montant et de la durée de mise en jeu, parmi d'autres critères. La
fonction de récompense choisie est prévue pour minimiser la variance, assurant que les gros stakers ne reçoivent pas de
récompenses disproportionnées par rapport aux autres. Les participants ne sont également pas soumis au
``facteur chance'' comme dans le minage par preuve de travail. Un tel schéma de récompense décourage également la
formation de groupes de minage ou de staking permettant ainsi une participation au réseau réellement décentralisée et
sans tiers de confiance.

\paragraph{Echanges atomiques de jetons}
A part assurer la sécurité centrale du système, le jeton \AVATokenName{} sert comme unité d'échange universelle. A
partir de là, la plateforme \AVAPlatformName{} sera capable de supporter les échanges atomiques de jetons (atomic swaps)
nativement permettant des échanges entre tous types d'actifs réellement décentralisés directement sur la plateforme.

\subsection{Gouvernance}
La gouvernance est critique pour le développement et l'adoption de toute plateforme car - comme avec d'autres types de
systèmes - \AVAPlatformName{} fera également face à une évolution naturelle et à des mises à jour. \AVATokenName{}
apporte une gouvernance sur la blockchain pour les paramètres critiques du réseau avec laquelle les participants ont la
possibilité de voter des changements au réseau et se mettre d'accord sur les décisions d'évolutions du réseau de manière
démocratique. Cela inclue les facteurs tels que le montant minimal de staking, le taux d'émission, ainsi que d'autres
paramètres économiques. Cela permet à la plateforme d'effectuer en pratique l'optimisation de paramètres dynamiques à
travers un oracle populaire. Cependant, contrairement à d'autres plateformes de gouvernance existantes,
\AVAPlatformName{} ne permet pas des changements illimités sur tous les aspects du système. A la place, seul un nombre
prédéfini de paramètres peut être modifié par gouvernance, rendant le système plus prévisible pour une sûreté de
fonctionnement accrue. De plus, tous les paramètres gouvernables sont sujets à des limites sur des périodes de temps,
introduisant de l'hystérésis, et assurant que le système demeure prévisible sur des petites périodes de temps.

Un procédé effectif pour trouver des valeurs globales acceptables pour les paramètres du système est critique pour les
systèmes décentralisés sans dépositaires. \AVAPlatformName{} peut utiliser son mécanisme de consensus pour élaborer un
système permettant à toute personne de proposer des transactions spéciales qui sont en essence des sondages au niveau
système. Tout noeud participant peut émettre de telles propositions.

Le taux de récompense nominal est un paramètre important qui affecte toutes les monnaies, qu'elles soient digitales ou
fiduciaires. Malheureusement les cryptomonnaies qui fixent ce paramètre peuvent faire face à de nombreux problèmes,
incluant la déflation ou l'inflation. Dans ce but, le taux de récompense nominal est soumis à la gouvernance dans un
cadre prédéfini. Cela permettra aux possesseurs de jetons de choisir si \AVATokenName{} doit finalement être limité,
illimité, ou même déflationnaire.

Les frais de transaction, dénotés par l'ensemble $\mathcal{F}$, sont également soumis à la gouvernance. En pratique
$\mathcal{F}$ est un tuple qui décrit les frais associés à diverses instructions et transactions. Finalement, les temps
et les montants de mise en jeu sont également modifiables par gouvernance. La liste de ces paramètres est définie en
Figure~\ref{fig:notation}.

\begin{figure}[hbtp]
\begin{framed}
\begin{itemize}
\item{$\Delta$} : Montant mis en jeu, dénominé en \AVATokenName. Cette valeur définit le montant minimal qui doit être
mis en jeu comme garantie avant de participer au système.
\item{$\delta_{min}$} : La durée minimale requise pour laquelle un noeud met en jeu sa particpation au système.
\item{$\delta_{max}$} : La durée maximale pendant laquelle un noeud peut engager une mise en jeu.
\item{$\rho: (\pi\Delta,\tau\delta_{min}) \rightarrow \mathbb{R}$} : La fonction du taux de récompense, aussi appelée
taux d'émission, qui détermine la récompense qu'un participant peut demander en fonction du montant qu'il a mis en jeu
pour un nombre $\pi$ de noeuds lui appartenant officiellement, sur une période de $\tau$ périodes de temps
$\delta_{min}$ consécutives, tel que $\tau\delta_{min} \leq \delta_{max}$.
\item{$\mathcal{F}$} : la structure des frais, qui est un ensemble de paramètres gouvernables liés aux frais et qui
spécifient les coûts de diverses transactions.
\end{itemize}
\end{framed}
\caption{Paramètres principaux non liés au consensus utilisés dans \AVAPlatformName{}. Toute notation est redéfinie à
partir de sa première utilisation.}
\label{fig:notation}
\end{figure}

En accord avec le principe de prédictibilité d'un système financier, la gouvernance dans \AVATokenName{} comporte de
l'hystérésis, ce qui veut dire que les changements de paramètres sont extrêmement dépendants de leur changements les
plus récents. Il y a deux limites associées à chaque paramètre gouvernable: le temps et la plage de valeurs. Une fois
un paramètre changé par une transaction de gouvernance, il devient très difficile de le changer à nouveau immédiatement
et par une différence importante. Ces contraintes de difficulté et de valeur se relâchent au fur et à mesure que le
temps passe depuis le dernier changement. Globalement, cela empêche le système de changer de manière drastique sur une
petite période de temps, permettant aux utilisateurs de prévoir avec certitude les paramètres du système à court terme,
tout en ayant un fort contrôle et une forte flexibilité sur le long terme.

\section{Discussion}
\label{section:discussion}
\subsection{Optimisations}
\subsubsection{Nettoyage}
Beaucoup de plateformes à base de blockchain, spécialement celles implémentant un consensus de Nakamoto comme Bitcoin,
souffrent d'une croissance perpétuelle de leur état. Celà est dû -- au niveau du protocole -- au fait qu'ils doivent
stocker l'historique entier des transactions. Cependant, pour qu'une blockchain puisse croître significativement, elle
doit être capable de nettoyer son historique ancien. C'est particulièrement important pour les blockchains à haute
performance comme \AVAPlatformName{}.

Le nettoyage (``pruning'' en anglais) est relativement simple dans la famille Snow*. Contrairement à Bitcoin (et autres
protocoles similaires) pour lequel le nettoyage n'est pas possible pour cause de contraintes algorithmiques, avec
\AVATokenName{} les noeuds n'ont pas besoin de conserver les parties profondes du DAG qui sont largement vérifiées. Ces
noeuds n'ont pas besoin de démontrer toute histoire ancienne auprès des nouveaux noeuds qui s'amorcent, et n'ont donc à
stocker que leur état courant, comment les soldes actuels et les transactions non traitées.

\subsubsection{Les types de clients}
\AVAPlatformName{} peut supporter trois types de clients différents: client d'archive, complet, et léger. Les noeuds
d'archives stockent l'historique complet du sous-réseau \AVATokenName{}, le sous-réseau de staking, et le réseau gérant
les contrats intelligents, jusqu'à la génèse de ceux-ci, ce qui veut dire que ces noeuds servent de noeuds d'amorçage
pour les nouveaux noeuds entrants. De plus ces noeuds peuvent stocker l'historique complet d'autres sous-réseaux pour
lesquels ils choisissent d'en être validateurs. Les noeuds d'archives sont typiquement des machines avec des grosses
capacités de stockage qui sont payées par d'autres noeuds lorsqu'ils téléchargent un état passé. Les noeuds complets
d'un autre côté participent à la validation, mais à la place de stocker l'historique complet ils gardent uniquement
l'état actif (ex: le jeu de sorties non dépensées en cours). Finalement, pour ceux qui nécessitent simplement
d'intéragir de manière sécurisée avec le réseau en utilisant la quantité de ressources la plus limitée,
\AVAPlatformName{} propose des clients légers qui peuvent prouver qu'une transaction a été acceptée sans avoir à
télécharger ou se synchroniser avec l'historique. Les clients légers se lancent dans les phase d'échantillonnage répété
du protocole pour assurer la sûreté de la validation et un consensus au niveau du réseau. En conséquent les clients
légers sur \AVAPlatformName{} offrent les même garanties de sécurité que les noeuds complets.

\subsubsection{Sharding}
Le sharding est un procédé consistant à partitionner différentes ressources système dans le but d'améliorer les
performances et réduire la charge. Il y a différents types de mécanismes de sharding. Dans le sharding réseau,
l'ensemble des participants est divisé en sous-réseaux séparés pour réduire la charge algorithmique; dans le sharding
d'état les participants s'accordent sur le stockage et la maintenance de sous-parties spécifiques de l'état global
uniquement; et en dernier le sharding de transactions pour lequel les participants se partagent le traitement des
transactions entrantes.

Dans \AVAPlatformNameFirstRelease{}, la première forme de sharding existe à travers le principe des sous-réseaux. Par
exemple, l'on pourrait lancer un sous-réseau de l'or et et un autre sous-réseau de l'immobilier. Ces deux sous-réseaux
peuvent exister entièrement en parallèle. Les sous-réseaux intéragissent seulement lorsqu'un utilisateur souhaite
acheter un bien immobilier en utilisant son or, à partir de ce moment \AVAPlatformName{} permettra de faire un
échange atomique entre les deux sous-réseaux.

\subsection{Préoccupations}
\subsubsection{Cryptographie post-quantique}
La cryptographie post-quantique a récemment retenu l'attention générale dû aux avancements dans le développement
d'ordinateurs et d'algorithmes quantiques. Le problème avec les ordinateurs quantiques est leur capacité à casser
certains des protocoles cryptographiques actuellement déployés, plus spécifiquement les signatures numériques. Le modèle
de réseau d'\AVAPlatformName{} autorise un nombre arbitraire de machines virtuelles lui permettant de supporter une
machine virtuelle résistante aux attaques quantiques avec un mécanisme de signature numérique approprié. Nous
anticipons plusieurs types de schémas de signatures numériques pouvant être déployés, incluant les signatures RLWE
résistantes aux attaques quantiques. Le mécanisme de consensus ne considère aucun type de cryptographie lourd pour son
fonctionnement principal. Au vu de cette conception, il est trivial de faire évoluer le système avec une nouvelle
machine virtuelle qui offre des primitives de cryptographie sécurisées face au quantique.

\subsubsection{Adversaires réalistes}
La papier Avalanche~\cite{avalanche} apporte des garanties très fortes en la présence d'un adversaire hostile puissant,
connu comme l'adversaire adaptatif aux tours dans le modèle point à point complet. En d'autres termes l'adversaire a un
accès complet à l'état de tous les nodes honnêtes \emph{à tout moment}, connaît les choix aléatoires de chaque node
honnête, peut également changer son propre état à n'importe quel moment, avant et après que le noeud honnête ait eu la
chance de changer son propre état. En pratique cette adversaire est tout-puissant, à l'exception de la capacité à
directement changer l'état d'un noeud honnête ou modifier la communication entre les noeuds honnêtes. Néanmoins, en
réalité, un tel adversaire est purement théorique sachant que l'implémentation d'un adversaire le plus puissant qu'il
soit est limité à des approximations statistiques de l'état du réseau. Par conséquent, en pratique nous nous attendons
à ce que les attaques correspondant au pire scénario soient difficiles à déployer.

\subsubsection{Inclusion et égalité}
Un problème courant dans les monnaies sans tiers de confiance se pose lorsque ``les riches deviennent de plus en plus
riches''. C'est une préoccupation valide, vu qu'un système à preuve d'enjeu mal implémenté peut en pratique permettre à
la génération de richesse d'être disproportionnellement attribuée au participant déjà détenteur du plus grand nombre de
jetons dans le système. Un exemple simple est celui des protocoles de consensus basés sur un leader, dans lesquels un
sous-comité ou un leader désigné collecte toutes les récompenses pendant son fonctionnement, et où la probabilité d'être
choisi pour collecter les récompenses est proportionnelle à la somme mise en jeu, engrangeant ainsi des effets
importants de récompense accumulée. De plus, dans les systèmes comme Bitcoin, il y a un phénomène où ``les plus gros
deviennent de plus en plus gros'' où les gros mineurs profitent d'un avantage par rapport aux plus petits en termes
de blocs orphelins moins nombreux et moins de travail perdu. Par contraste, Avalanche emploie une distribution
égalitaire de l'émission: chacun des participants au protocole de mise en jeu est récompensé de manière équitable et
proportionnelle à la somme mise en jeu. En permettant à un grand nombre de personnes de participer en premier plan
au staking, Avalanche peut supporter jusqu'à des millions d'individus participant de manière égale au staking. Le
montant minimal requis pour participer au protocole sera soumis à la gouvernance, mais il sera initialisé à une valeur
faible pour encourager une large participation. Cela implique également que la délégation ne soit pas nécessaire pour
participer avec une faible part.

\section{Conclusion}
\label{section:conclusion}
Dans ce papier, nous avons décrit l'architecture de la plateforme \AVAPlatformName{}. Comparée à d'autres plateformes à
ce jour, qui exécutent soit des protocoles classiques et donc intrinsèquement non évolutifs, ou utilisent un consensus
Nakamoto qui est inefficient et impose des coûts d'opération élevés, la plateforme \AVAPlatformName{} est légère, rapide,
évolutive, sûre, et efficiente. Le jeton natif qui sert à sécuriser le réseau et payer pour les différents coûts
d'infrastructure est simple et propose une compatibilité ascendante. \AVATokenName{} a la capacité à surpasser les
autres propositions dans l'obtention d'un niveau de décentralisation élevé, la capacité à résister aux attaques, et le
passage à l'échelle de millions de noeuds sans aucun quorum ou comité elu, et donc sans imposer aucune limite à la
participation.

En plus du moteur de consensus, \AVAPlatformName{} innove à tous les étages et introduit des idées simples mais
importantes dans la gestion des transactions, la gouvernance, et un tas d'autres composants que l'on ne retrouve dans
aucune autre plateforme. Chaque participant au protocole aura une voix pour influencer la manière avec laquelle le
protocole évolue \emph{à tout moment}, rendu possible par un mécanisme de gouvernance puissant. \AVAPlatformName{}
permet un haut niveau de customisation, permettant une connexion plug-and-play presque instantanée avec les blockchains
existantes.

\bibliography{ava-platform}
\bibliographystyle{splncs04}
\end{document}
